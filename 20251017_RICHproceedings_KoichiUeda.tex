%% Copyright 2007-2025 Elsevier Ltd
%% 
%% This file is part of the 'Elsarticle Bundle'.
%% ---------------------------------------------
%% 
%% It may be distributed under the conditions of the LaTeX Project Public
%% License, either version 1.3 of this license or (at your option) any
%% later version.  The latest version of this license is in
%%    http://www.latex-project.org/lppl.txt
%% and version 1.3 or later is part of all distributions of LaTeX
%% version 1999/12/01 or later.
%% 
%% The list of all files belonging to the 'Elsarticle Bundle' is
%% given in the file `manifest.txt'.
%% 
%% Template article for Elsevier's document class `elsarticle'
%% with numbered style bibliographic references
%% SP 2008/03/01
%% $Id: elsarticle-template-num.tex 272 2025-01-09 17:36:26Z rishi $
%%
% \documentclass[preprint,1p,times]{elsarticle} %査読時にはこちらを使用。
\documentclass[5p,times]{elsarticle} %NIMで最終的に何ページになるかは確認するには、こちらを使用。(最終提出もこっち?)
\usepackage{lineno}
\usepackage{graphicx}
\usepackage{hyperref} % プリアンブルに追加

%% Use the option review to obtain double line spacing
%% \documentclass[authoryear,preprint,review,12pt]{elsarticle}

%% Use the options 1p,twocolumn; 3p; 3p,twocolumn; 5p; or 5p,twocolumn
%% for a journal layout:
%% \documentclass[final,1p,times]{elsarticle}
%% \documentclass[final,1p,times,twocolumn]{elsarticle}
%% \documentclass[final,3p,times]{elsarticle}
%% \documentclass[final,3p,times,twocolumn]{elsarticle}
%% \documentclass[final,5p,times]{elsarticle}
%% \documentclass[final,5p,times,twocolumn]{elsarticle}

%% For including figures, graphicx.sty has been loaded in
%% elsarticle.cls. If you prefer to use the old commands
%% please give \usepackage{epsfig}

%% The amssymb package provides various useful mathematical symbols
\usepackage{amssymb}
%% The amsmath package provides various useful equation environments.
\usepackage{amsmath}
%% The amsthm package provides extended theorem environments
%% \usepackage{amsthm}

%% The lineno packages adds line numbers. Start line numbering with
%% \begin{linenumbers}, end it with \end{linenumbers}. Or switch it on
%% for the whole article with \linenumbers.
%% \usepackage{lineno}

\usepackage{cleveref}
\usepackage{makecell} % プリアンブルに追加
\usepackage{booktabs}
\usepackage{float} % プリアンブルに追加


\journal{Nuclear Instruments and Methods in Physics Research, Section A}

\begin{document}

\begin{frontmatter}

\title{Development of a Gaseous Photomultiplication Based Cherenkov Detector Targeting Picosecond Time Resolution}

%% use optional labels to link authors explicitly to addresses:
%% \author[label1,label2]{}
%% \affiliation[label1]{organization={},
%%             addressline={},
%%             city={},
%%             postcode={},
%%             state={},
%%             country={}}

\author{Koichi Ueda\textsuperscript{a}, Kodai Matsuoka\textsuperscript{a,b,c}, Kenji Inami\textsuperscript{a,b}, Ryogo Okubo\textsuperscript{d}, Simone Garnero\textsuperscript{e}} %% Author name

%% Author affiliation
% \affiliation[a]{organization={},%Department and Organization
%             addressline={b}, 
%             city={c},
%             postcode={d}, 
%             state={e},
            % country={f}}

\affiliation[a]{{Graduate School of Science, Nagoya University, Furo-cho, Chikusa-ku, Nagoya, 464-8602, Japan}}
\affiliation[b]{{High Energy Accelerator Research Organization (KEK), 1-1 Oho, Tsukuba, Ibaraki, 305-0801, Japan}}
\affiliation[c]{{Kobayashi-Maskawa Institute for the Origin of Particles and the Universe (KMI), Nagoya University, Furo-cho, Chikusa-ku, Nagoya, 464-8602, Japan}}
\affiliation[d]{{The Graduate University for Advanced Studies (SOKENDAI), Hayama, Miura District, Kanagawa, 240-0193, Japan}}
\affiliation[e]{{INFN Sezione di Trieste, Via Valerio 2, 34127 Trieste, Italy}}
\affiliation[f]{{University of Bologna, Via Zamboni 33, 40126 Bologna, Italy}}

\begin{abstract}
To meet the demands of future large-scale experiments for Cherenkov detectors with high time resolution, large area, and low cost, we are developing a Gaseous Photomultiplier (GasPM). Previous studies demonstrated a single-photon time resolution of FWHM $= 58.9 \pm 2.6$ ps using a pico-second pulse laser. Furthermore, a GasPM achieved a resolution of FWHM $= 253.8$ ps for Cherenkov light generated by an electron beam. To achieve a better resolution, we have modified the GasPM by increasing the electric field and using a thicker radiator. We expected these modifications would result in a higher detection efficiency, an increased gain, a faster rise-time, and an improved time resolution. The performance was evaluated using a 5 GeV/c electron beam at the KEK PF-AR test beam line. The number of detected Cherenkov photons was lower than expected. The signal pulses had a fast rise time similar to the results in the laser test. Although the time resolution improved compared to the previous beam test, it did not reach the value in the laser test. We attribute these results primarily to a potential degradation of the photocathode's quantum efficiency and the effects of photon feedback.
\end{abstract}


%%Graphical abstract
% \begin{graphicalabstract}
% \includegraphics{grabs}
% \end{graphicalabstract}

%%Research highlights
% \begin{highlights}
% \item Research highlight 1
% \item Research highlight 2
% \end{highlights}

% Keywords
\begin{keyword}
Gaseous detectors, Resistive-plate chambers, Cherenkov detectors, Instrumentation
and methods for time-of-flight (TOF) spectroscopy, Photon detectors for UV, visible and IR photons
(gas), Timing detectors
\end{keyword}

\end{frontmatter}
% \tableofcontents
\newpage
\linenumbers % 行番号表示を開始

%% Use \section commands to start a section
\section{This is a test section}
\section{Introduction}
\label{introduction}

To develop Cherenkov detectors with high time resolution, a large area, and a low cost for future large-scale experiments, we are developing a Gaseous Photomultiplier (GasPM). The GasPM is a photon detector that combines a Resistive Plate Chamber (RPC) with a photocathode. It operates as a Cherenkov detector by detecting Cherenkov photons produced in its entrance window
% (\cref{gaspm_schematic}, \cref{GasPM_picture}).
(\cref{gaspm_schematic_and_picture}).

% \begin{figure}
%   \centering
%   \includegraphics[width=0.3\textwidth]{figures/gaspm_graphic.png}
%   \caption{Schematic of the GasPM.}
%   \label{gaspm_schematic}
%   \end{figure}
%   \begin{figure}
%     \centering
%     \includegraphics[width=0.35\textwidth]{figures/gaspm_picture.jpg}
%   \caption{Picture of the GasPM prototype. The dark square at the center is the sensitive area ($26\times26~\mathrm{mm}$).}
%   \label{GasPM_picture}
%   \end{figure}
\begin{figure}
    \centering
    \includegraphics[width=0.5\textwidth]{figures/gaspm_schematic_and_picture.jpg}
    \caption{(Left) A schematic of the GasPM operating as a Cherenkov detector. (Right) A picture of the GasPM prototype. The dark square at the center is the sensitive area ($26\times26~\mathrm{mm}$).}
    \label{gaspm_schematic_and_picture}
\end{figure}
Using a pico-second pulse laser, a GasPM with a LaB$_6$ photocathode and a mixture of R134a and SF$_6$ gas achieved a single-photon time resolution of FWHM $= 58.9 \pm 2.6$ ps ($\sigma=25 \pm 1.1$ ps)\cite{matsuoka2025}. Using an electron beam at the KEK PF-AR test beamline, a GasPM with a CsI photocathode and an MgF$_2$ window and a low electric field achieved a time resolution of FWHM $= 253.8$ ps in 2023\cite{okubo2024}. In both tests, events with pulse overlap
% (\cref{pulseoverlap})
due to photon feedback had a worse time resolution. Photon feedback is a phenomenon in which photons generated during the avalanche process cause additional avalanches.
% \begin{figure}
%   \centering
%   \includegraphics[width=0.4\textwidth]{figures/photonfeedback_pulseshape.png}
%     \caption{An example waveform with pulse overlap due to photon feedback. the second pulse is caused by photon feedback.}
%     \label{pulseoverlap}
% \end{figure}
    % \subsection{Development plans}
Our goal is to achieve a time resolution of FWHM $\approx 20$ ps for charged particles. To achieve this, we plan to increase the gap electric field, enhance the number of detected photons by using a thicker radiator or a photocathode with a higher quantum efficiency, and improve the read-out sampling rate from 5 GSPS to 10 GSPS to better separate the photon feedback pulse from the main pulse.

\section{Performance of the GasPM prototype}
\subsection{GasPM setup}
To achieve the goal, we modified the GasPM's electric field and the radiator's thickness (\cref{tab:gaspm_setup}).
\begin{table}[htbp]
  \centering
  \caption{GasPM setup parameters.}
  \label{tab:gaspm_setup}
  \resizebox{0.45\textwidth}{!}{
  \begin{tabular}{ l l l }
    \toprule
    Parameter & 2023 test value & 2025 test value \\
    \midrule
    Electric field & \makecell[l]{$-140$ kV/cm \\ ($-2.8$ kV / 200 µm)} & \makecell[l]{$-187$ kV/cm \\ ($-2.8$ kV / 150 µm)} \\
    \addlinespace
    \makecell[l]{Radiator (Window) \\ thickness} & 2.4 mm & 5.0 mm \\
    \bottomrule
  \end{tabular}
  }
\end{table}
We expected that these modifications would result in a higher detection efficiency, an increased gain, a faster rise-time, and an improved time resolution.
The GasPM with a photocathode can produce both ionization signals, when operating as an RPC, and Cherenkov photon signals, when detecting Cherenkov light. On the other hand, the GasPM without a photocathode operates as an RPC and produces only ionization signals.
\subsection{Measurement setup}
We used an electron beam with an energy of 5 GeV at the KEK PF-AR test beam-line. We placed the GasPM, a Micro Channel Plate PMT (MCP-PMT) with a quartz radiator as a time reference, an MPPC-array with an acrylic radiator to veto multiple electron events, and two PMTs with plastic scintillators as trigger counters in the beam line (\cref{beamtest_setup}). Signals from the GasPM, the MCP-PMT, and the trigger counters were read out using a DRS4 Evaluation Board (Paul Scherrer Institut), which has a sampling rate of 5 GSPS and a time resolution of FWHM $= 32.3 \pm 0.2$ ps.
\begin{figure}
    \centering
    \includegraphics[width=0.4\textwidth]{figures/beamtest_setup.png}
    \caption{A schematic of the beam test setup at KEK PF-AR test beam-line.}
    \label{beamtest_setup}
\end{figure}
% \subsection{Results}

\subsection{Signals}
As shown as \cref{pulseshape}, we successfully observed signal pulses.
\begin{figure}
    \centering
    \includegraphics[width=0.4\textwidth]{figures/result_pulse_shape.png}
    \caption{Examples of signal pulses.}
    \label{pulseshape}
\end{figure}

\subsection{Detection efficiency}
We set the pulse height thresholds to separate the signal from the noise (\cref{height}).
We determined the beam detection probability ($P$) and calculated the average number of avalanches ($N_{\text{avalanche}}$) as $-\ln(1-P)$. We obtained the number of detected Cherenkov photons by subtracting the $N_{\text{avalanche}}$ without a photocathode from the $N_{\text{avalanche}}$ with a photocathode. As shown in \cref{tab:my_table_makecell}, the number of detected Cherenkov photons did not increase.
% \begin{table}[htbp]
%   \centering
%   \caption{Signal types and their generation possibilities with and without a photocathode.}
%   \label{signal_types}
%   \begin{tabular}{ l c c } % Removed vertical lines |
%     \toprule % Replace \hline
%     Signal type & w/o photocathode & w/ photocathode \\
%     \midrule % Replace \hline
%     Ionization signal & \checkmark & \checkmark \\
%     Cherenkov photon & $\times$ & \checkmark \\ % $\times$ for the cross mark
%     \bottomrule % Replace \hline
%   \end{tabular}
% \end{table}
\begin{figure}[h]
    \centering
    \includegraphics[width=0.4\textwidth]{figures/result_height.png}
    \caption{The pulse height distribution of the GasPM with and without a photocathode.}
    \label{height}
\end{figure}
\begin{table}[H]
  \centering
  \caption{Comparison of detection parameters between 2023 and 2025.}
  \label{tab:my_table_makecell}
  \setlength{\tabcolsep}{3pt} % 列間のスペースを少し詰める (値は調整)
  % {\footnotesize
  \resizebox{0.5\textwidth}{!}{
  \begin{tabular}{c c c c c c}
    \toprule
    & \multicolumn{2}{c}{\textbf{w/o photocathode}} & \multicolumn{3}{c}{\textbf{w/ photocathode}} \\
    \cmidrule(lr){2-3} \cmidrule(lr){4-6}
    & \makecell{Beam detection \\ efficiency ($P$)} % <-- ここで改行
    & \makecell{$N_{\text{avalanche}}$ \\ ($-\ln(1-P)$)} % <-- ここで改行
    & $P$
    & \makecell{$N_{\text{avalanche}}$} % <-- ここで改行
    & \makecell{\textbf{Number of} \\ \textbf{detected} \\ \textbf{Cherenkov photons}} \\ % <-- 3行に改行
    \midrule
    \textbf{2023 test} & $77.2 \pm 0.5\%$ & $1.48 \pm 0.03$ & $96.5 \pm 0.3\%$ & $3.34 \pm 0.06$ & $\mathbf{1.86 \pm 0.07}$ \\
    \textbf{2025 test} & $63 \pm 2\%$ & $1.00 \pm 0.05$ & $93.4 \pm 0.4\%$ & $2.71 \pm 0.07$ & $\mathbf{1.71 \pm 0.09}$ \\
    \bottomrule
  \end{tabular}
    }
\end{table}

\subsection{Rise-time and time resolution}
To reduce the effects of pulse overlap due to photon feedback, we used a specialized timing algorithm (\cref{timing_algorithm}). We fit the rising edge of the pulse with a polynomial function. We defined the signal timing ($T$) as the time when the slope of the fitted curve reaches the local maximum. We defined the rise-time as $(T-T')\times2$ ($T'$: the time at $20\%$ of the voltage at $T$).
% \cref{timing_algorithm} shows an example of the fitting result.
As shown in \cref{risetime_overlaid_waveforms}, the rise-time is faster than that in the 2023 test and similar to that in the laser test.
\cref{time_resolution} shows the time resolution of the TOF measured by
the GasPM and the MCP-PMT. We fit the distribution by two Gaussian functions. The resolution is FWHM $= 178.1$ ps.
\begin{figure}[h]
    \centering
    \includegraphics[width=0.35\textwidth]{figures/timing_algorithm.png}
    \caption{An Example of the timing algorithm fitting result. The upper figure shows a raw waveform and the fitted curve. The lower figure shows the slope of the fitted curve.}
    \label{timing_algorithm}
\end{figure}
\begin{figure}[H]
    \centering
    \includegraphics[width=0.5\textwidth]{figures/risetime_overlaid_waveforms.png}
    \caption{(left) Rise-time distribution. (right) Overlaid 3 waveforms.}
    \label{risetime_overlaid_waveforms}
\end{figure}
\begin{figure}[H]
    \centering
    \includegraphics[width=0.4\textwidth]{figures/timeresolution.png}
    \caption{A time resolution of the TOF measured by the GasPM and the MCP-PMT for the 5 GeV $e^-$ beam.}
    \label{time_resolution}
\end{figure}
% \subsection{Rise time}
% \subsection{Time resolution}
\section{Discussion and future plans}
The number of detected Cherenkov photons was lower than our expectation. This can be attributed to a lower quantum efficiency.
Although a stronger electric field improved the resolution, the resolution was worse than expected. A potential cause of the worse resolution could be the photon feedback.
To improve the performance of the GasPM, we plan to investigate the cause of the lower quantum efficiency. Additionally, we will analyze the data from the 10 GSPS digitizer to see if the higher sampling rate helps mitigate the photon feedback effect.

\section{Conclusion}
For future large experiments, we are developing the GasPM. We conducted the beam tests and found that the major issue in achieving a high time resolution could be the photon feedback. We will address it using a digitizer with a higher sampling rate. We will also try to improve the time resolution by addressing the low quantum efficiency.




%% The Appendices part is started with the command \appendix;
%% appendix sections are then done as normal sections
% \appendix
% \section{Example Appendix Section}
% \label{app1}

% Appendix text.

% %% For citations use: 
% %%       \cite{<label>} ==> [1]

% %%
% Example citation, See \cite{lamport94}.

%% If you have bib database file and want bibtex to generate the
%% bibitems, please use
%%
%%  \bibliographystyle{elsarticle-num} 
%%  \bibliography{<your bibdatabase>}

%% else use the following coding to input the bibitems directly in the
%% TeX file.

%% Refer following link for more details about bibliography and citations.
%% https://en.wikibooks.org/wiki/LaTeX/Bibliography_Management

\begin{thebibliography}{00}

%% For numbered reference style
\bibitem{matsuoka2025}
K. Matsuoka et al., "Demonstration of a 25-picosecond single-photon time resolution with gaseous photomultiplication",
\href{https://www.sciencedirect.com/science/article/pii/S0168900223003686?via%3Dihub}{Nucl. Instrum. Methods Phys. Res. Sect. A, 1053, 168378 (2023).}
\bibitem{okubo2024}
R. Okubo et al., "Development of a picosecond-timing Cherenkov detector using gaseous photomultiplification," 
\href{https://iopscience.iop.org/article/10.1088/1748-0221/20/08/C08017}{JINST 20 (2025) C08017, Proceedings of the 6th International Workshop on new Photon-Detectors (PD24), Simon Fraser University, Harbour Centre, Vancouver, BC, Canada, November 19–21, 2024.}

\end{thebibliography}
% \end{comment}
\end{document}

\endinput
%%
%% End of file `elsarticle-template-num.tex'.
